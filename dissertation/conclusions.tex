\chapter{Conclusion}
The project proceeded better than was expected, resulting in three extensions being implemented. It was shown to be a statistically significant improvement over existing tools in terms of learning curve. This was the result of numerous design iterations, and a lot of code to develop the custom charting system that best matched the project needs.

\section{Comparison with Requirements}
The requirements originally listed in the preparation chapter(\autoref{sec:Requirements}) are restated below for easy comparison, to show that they were met and exceeded.

%TODO To restate or not?

The system must allow the user to:
	\begin{enumerate}[label=\bfseries Core \arabic*]
		\item Visualise data they have stored in common file formats.
		\item Specify the type of chart they want by drawing a likeness of it on screen using a stylus.
		\item Bind the data to the chart using an interface that makes it clear how the data is being used to generate the graphics.
		\item Specify visual properties of the chart, such as size and position, through the sketches.
		\item Manipulate the visual appearance of the created chart.
	\end{enumerate}
	
	These core requirements were all met.
	In addition, time permitting, the system may:
	\begin{enumerate}[label=\bfseries Extension \arabic*]
		\item Transform user-drawn sketches to resemble and overlap the formal chart elements they generated, in order to show the link between the sketch and the formal elements visually.
		\item Mirror any manipulations applied to the formal layer back to the user drawn sketches, in order to keep the visuals of the formal and sketch layers in synchronisation.
		\item Allow users to undo actions by erasing sketches, and remove the corresponding formal elements without throwing errors.
		\item Allow users to manipulate any property of chart elements, not just one, so that the domain of visualisations they can create is infinite. For example, allow them to bind not just the height of bars in a bar chart to data, but also their width and colour. 
		\item Analyse the data and infer properties that may allow it to automatically suggest properties of the chart, such as which field belongs on which axis, or whether a data series should be log scale or linear scale.
		\item Support exporting the chart as a Microsoft Chart object that can be embedded as a dynamic object in Microsoft Office files, not just as a raster image.
	\end{enumerate}
	
	Extensions 1, 2 and 3 were implemented, resulting in an even more easily learnable design. Extension 4 could have been built since most of the technical platform required already existed, but was decided against in order to keep the application simple for users to understand, and make it possible to make charts quickly. The remaining two extensions were of sufficient technical complexity to not be feasible to complete within the time constraints.	
	
	Some usability goals were also specified:
	\begin{enumerate}[label=\bfseries Usability \arabic*]
		\item Users must be able to create charts at least as quickly as they can using current charting systems.
		\item Users must be able to build a mental model of the software's behaviour within 2 uses of it. They should thus be able to accurately predict the consequences of any action taken within the software.
		\item Changes to the visualisation must occur through directly manipulating the visual representation of the chart, rather than through disconnected User Interface widgets.
		\item The user must be able to easily try out changes to the visualisation, see what the resultant chart would look like, and undo them if needed.
	\end{enumerate}
	
	Anecdotally, users were able to create charts within minutes, and made positive comments about the perceived speed. The user study demonstrated that they had built up a mental model by correctly applying the understanding they had gained from one task, to perform a second, similar task quicker. All changes to the chart are made directly on the chart, there is minimal other UI. The second half of the user study demonstrated that the application could encourage exploration, since users made the same modification to their chart significantly faster in Sketchography than the leading charting tool, Microsoft Excel.
	
\section{Future Work}
While the goals of the project were met, the exploratory design phase as well as the user testing generated a lot of ideas that weren't feasible to implement within the time scale.

The primary improvement that would make this tool more practical would be to enable exporting the chart, as an image, or better yet, as a Microsoft Office Chart object that can be embedded as a vector object in office applications.

Additionally, the technical platform built could be extended to support many more types of charts. Given the small amount of data collected to train the classifier, adding recognition support for many more chart elements could have lowered accuracy, which would be detrimental to the usability. However, with more time, more data could be collected, for a variety of chart types.

More features could be added to the charting component. It might, for example, apply heuristics such as shifting the axes slightly to make them align with each other, or choose round numbers for axis labels rather than precisely splitting up the range of values.

In the short term, I will be packaging the \texttt{SketchChart} component for distribution for others to use in their applications. In this project, I have broken new ground by creating a third-party User Interface component for .NET applications that accepts sketch input. While I have built a powerful sketch recognition platform, and accompanying custom charting component, there are a lot more features that could be added. The project was designed in an extensible manner to allow the domain of sketches to be expanded or changed to another one entirely. Hence, I will be making the repository public and adding necessary documentation to invite contributions from other developers, and to let the code support other use cases that would benefit from the direct manipulation and liveness.

