\chapter{Evaluation}

\begin{verbatim}
> t.test(Task1, Task2, paired = TRUE)

	Paired t-test

data:  Task1 and Task2
t = 5.2186, df = 9, p-value = 0.0005502
alternative hypothesis: true difference in means 
is not equal to 0
95 percent confidence interval:
 23.34058 59.05942
sample estimates:
mean of the differences 
                   41.2 
\end{verbatim}

The test is valid if the data follows a normal distribution. For each of the tasks, below is a Q-Q plot comparing the actual values (points) to the theoretical normal (line), and the results of the Shapiro Wilk test of normality.

\begin{figure}[H]
		\centering
		\begin{subfigure}[b]{\textwidth}
			\includegraphics[width=\textwidth]{figure/task1qq.png}
		\end{subfigure}
		\begin{subfigure}[b]{\textwidth}
			\includegraphics[width=\textwidth]{figure/task2qq.png}
		\end{subfigure}
		\caption{Normality Tests for Creation Tasks}
	\end{figure}

\begin{tabular}{l l l}
Dataset & W & p-value \\
Task 1 & 0.8435 & 0.04854 \\
Task 2 & 0.8317 & 0.03511 \\
\end{tabular}



Similar evidence that people took longer to change the bar height in Excel than in Sketchography.

\begin{verbatim}
> t.test(Sketch, Excel, paired = TRUE)

	Paired t-test

data:  Sketch and Excel
t = -4.5484, df = 9, p-value = 0.001389
alternative hypothesis: true difference in means 
is not equal to 0
95 percent confidence interval:
 -83.55241 -28.04759
sample estimates:
mean of the differences 
                  -55.8 
\end{verbatim}

\begin{figure}[H]
		\centering
		\begin{subfigure}[b]{\textwidth}
			\includegraphics[width=\textwidth]{figure/sketchqq.png}
		\end{subfigure}
		\begin{subfigure}[b]{\textwidth}
			\includegraphics[width=\textwidth]{figure/excelqq.png}
		\end{subfigure}
		\caption{Normality Test for Modification Tasks}
	\end{figure}


\begin{tabular}{l l l}
Dataset & W & p-value \\
Sketch & 0.9426 & 0.3966 \\
Excel & 0.8943 & 0.1896 \\
\end{tabular}