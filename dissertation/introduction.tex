\chapter{Introduction}
\section{Overview}
This project is an exploration of Human Computer Interface concepts governing the interactions of users with tools that let them explore and visualise data.

The design of currently available charting tools were constrained by the input devices available previously: mouse and keyboard. Thus, they usually allow graph generation through one of two interfaces:
\begin{enumerate}
\item A series of dialog boxes and wizards to walk the user through a number of choices.
\item Writing code that is interpreted to process data and generate graphics.
\end{enumerate}
This paper describes a different interface, which allows the user to sketch a subset of a chart on their computer touch screen like they would on paper. The hypotheses are that this interface is more learnable over time, and that it allows easier modification and exploration of visualisations compared to other charting applications. Both these hypotheses were investigated through a user study.

The end result is a proof-of-concept charting application that works as below:
\begin{enumerate}
%TODO Insert screenshots for each step
\item The user imports data from a Microsoft Excel file.
\item They sketch a rough indication of a chart.
\item They drag the data onto elements of the chart to actually bind the data to the chart. 
\item The tool then creates a 'formal' chart.
\item The tool transforms the user's original sketch to more closely match the formal chart, making the mapping between sketch and formal chart elements evident to the user.
\item Any changes on either the sketch or formal chart is fed through to the other view. For example, erasing the a sketched bar removes a data series from the formal bar.
\end{enumerate}

\section{Background}
Sketching inputs have been studied for a while as more natural interfaces to computers, especially for graphics-related tasks \citep{Sutherland1964}.
\section{Objectives}