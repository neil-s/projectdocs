\chapter{Introduction}
\section{Motivation}
This project is an exploration of Human Computer Interface concepts governing the interactions of users with tools that let them explore and visualise data.

The design of most charting tools is driven by the choice of interface: the mouse and keyboard. Thus, they usually allow graph generation through one of two means:
\begin{enumerate}
\item Configuring a chart template through a number of wizards and dialogue boxes.
\item Manually writing code to generate chart graphics.
\end{enumerate}

%TODO Should I just remove the first sentence below entirely, and replace it with the one below it?
Due to these design constraints, these solutions take users a relatively long time to learn how to use. 

Users need to familiarise themselves with how the choices in the wizard, or the commands in the language, translate to graphics.

A better solution is to use a metaphor to a system users have already learnt to use - drawing using pen and paper. Such a system would benefit from matching the users' mental model. 

Additionally, there is a long lag between the users expressing their intention in current systems, and seeing the results of their changes after they close the configuration dialogue or compile and re-run the code. A better solution would exhibit 'liveness' by immediately accommodating users' changes.

\section{Background and Related Work}
Sketching inputs have been studied since the 1960s \citep{sutherland_sketch_1964} as more natural interfaces to computers, especially for graphics-related tasks, compared to indirections like the mouse and keyboard. This has largely been motivated by the widely recognised importance of interaction to Information Visualisation (InfoVis) \citep{lee_beyond_2012}. 

%TODO Overlap between this and the design section. Doesn't talk about unfinishedness encouraging exploration.

Additionally, the metaphor of sketching on paper can encourage exploratory work due to the ease of creating changes and visually expressing what sort of change one is trying to make, by minimising the gap between a person's intent and the execution of the intent, or as \cite{norman_user_1986} call it, the `Gulf of Execution'.

Meanwhile, there has been increasing adoption of touch-enabled phones and multi-touch slates amongst the general public, demonstrating people's affection for what have been referred to as Natural User Interfaces \citep{lee_beyond_2012}.

%TODO include this?
Much more related work is discussed later, in the context of design decisions for this project.

\section{Project Description}
This paper describes a different interface, which allows the user to sketch a subset of a chart on their computer touch screen like as would on paper. The hypotheses are that:

\begin{enumerate}
\item[H1] This interface is more 'learnable' over time
\item[H2] It encourages exploratory data visualisation creation by making modification easier
\end{enumerate}

when compared to other charting applications. Both these hypotheses were investigated through a user study.
%TODO How to structure the above to be complete sentences.

The end result is a proof-of-concept charting application that works as below:
\begin{enumerate}
%TODO Insert screenshots for each step
\item The user imports data from a Microsoft Excel file.
\item They sketch a rough indication of a chart.
\item They drag the data onto elements of the chart to actually bind the data to the chart. 
\item The tool then creates a 'formal' chart.
\item The tool transforms the user's original sketch to more closely match the formal chart, making the mapping between sketch and formal chart elements evident to the user.
\item Any changes on either the sketch or formal chart is fed through to the other view. For example, erasing the a sketched bar removes a data series from the formal bar.
\end{enumerate}

